\Conclusion % заключение к отчёту

В результате выполнненных в ходе дипломного проектирования работ была рассмотрена структура нового противотанкового комплекса на базе ударно-разведывательного БПЛА, а также проведена разработка ПТУР для этого комплекса.
Для этого был сформирован облик комплекса, определена схема его функционирования, спроектирован ПТУР, подробно рассмотрен алгоритм наведения ПТУР.
Рассмотренный комплекс позволяет решить такие задачи как:
\begin{itemize}
 \item Повышение эффективности применения войск в тактических и оперативно-тактических операциях ВС РФ и других силовых ведомств РФ путем внедрения современных беспилотных средств вооруженной борьбы;
 \item Частичная компенсация дефицита боевых возможностей перспективной авиационной группировки за счет относительно низкой стоимости жизненного цикла комплекса;
 \item Повышение потенциала сил сдерживания за счет более широкого использования составляющей с неядерными средствами поражения;
 \item Уменьшение риска, которому подвержен личный состав войск во время вооруженного конфликта.
\end{itemize}

Помимо проектирования собственно комплекса был создан мощный математический аппарат для динамического моделирования, позволяющий выполнять предварительную оценку эффективности ПТУР при пуске с воздушных носителей при их проектировании.
После заверешения работ над комплексом данный аппарат можно использовать для аналогичных задач при проектировании новых образцовв ПТУР, а также корректируемых мин и авиабомб, а также для изучения показателей эффективности уже существующих образцов ПТУР.
