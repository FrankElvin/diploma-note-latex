\chapter{ТЕХНОЛОГИЧЕСКАЯ ЧАСТЬ}
\label{cha:ch_3}

Технологическая часть дипломного проекта заключается в разработке технологического процесса производства детали <<Днище двигательной установки>>

\section{Общая часть}

\subsection{Назначение детали}
Деталь является передним днищем двигателя, препятствующим прорыв продуктов сгорания твердого топлива в сторону вычислительного блока управляемой ракеты. Деталь должна держать 14 атмосфер в течении 4 секунд во время работы РДТТ на режиме. Днище соединяется резьбой с обечайкой двигателя, корпус ракеты соединяется с деталью четырьмя болтами. Также в отверстие в детали вкручивается воспламенитель двигателя твердого топлива.
Предъявляются требования к герметичности соединения детали с обечайкой двигателя и с воспламенителем. Также к детали предъявляются требования по массе с целью увеличения массы полезной нагрузки управляемой ракеты.

Материал детали - сталь 33Х3СНМВФА (сплав СП-33).

\subsection{Материал детали и его свойства}
Сталь 33Х3СНМВФА относится к классу легированной конструкционной стали. Сталь широко используется для изготовления поковок. Еще из нее производят цельнокатаные кольца, служащие для производства разнообразных деталей энергетического и тяжелого машиностроения.
Химический состав стали (в \%) представлен в таблице \ref{tab:techno_steel}.
\begin{table}[h]
	\begin{center}
		\caption{}
		\begin{tabular}{|l|l|l|l|l|l|l|l|}
		\hline
C & Si & Mn & Ni & S & P & Cr & Cu \\ \hline
0,28 - 0,34 & 0,9 - 1,2 & 0,8 - 1,1 & до   0,3 & до   0,025 & до   0,025 & 0,8 - 1,1 & до   0,3 \\ \hline
		\end{tabular}
		\label{tab:techno_steel}
	\end{center}
\end{table}

Физические свойства стали 33Х3СНМВФА:

твердость материала, HB  = 65 МПА.

\begin{table}[h]
	\begin{center}
		\caption{}
		\begin{tabular}{|l|l|l|l|l|l|l|l|}
		\hline
T &  $E \cdot 10^5 $ & $\alpha \cdot 10^6 $ & l  &   $\rho$ & C & $R \cdot 10^6 $   \\ \hline
Град & МПа & 1/Град & Вт/(м$\cdot$град) & кг/м3 & Дж/(кг$\cdot$град) & Ом$\cdot$м \\ \hline
20 & 2,15 &  & 38 & 7850 &  & 210 \\ \hline
100 & 2,11 & 11,7 & 38 & 7830 & 496 & \\ \hline
		\end{tabular}
		\label{tab:techno_steel}
	\end{center}
\end{table}

T - температура, при которой получены данные свойства, °C;

E - модуль упругости первого рода, МПа;

$\alpha$ - коэффициент температурного (линейного) расширения (диапазон 20° - T), 1/°C;

l - коэффициент теплопроводности (теплоемкость материала), Вт/(м$\cdot$°C);

$\rho$ - плотность материала, кг/$\text{м}^3$;

C - удельная теплоемкость материала (диапазон 20° - T), Дж/(кг$\cdot$°C);

R - удельное электросопротивление, Ом$\cdot$м.

\subsection{Выбор вида и метода получения заготовки}
Исходя из конструкции изделия и годового объема выпуска (мелкосерийное производство), для детали <<Днище двигательной установки>> целесообразно использовать штампованную заготовку. Это позволит повысить коэффициент использования материала и снизить объем механической обработки.

Учитывая опыт создания подобных деталей, требования к прочностным свойствам детали и механические свойства материала, для получения заготовки был выбран метод горячей объемной штамповки.

\subsection{Расчет припусков на механическую обработку}
Припуск – слой материала, назначаемый для компенсации погрешностей, возникающих в процессе изготовления детали, в целях обеспечения заданного ее качества. Различают минимальные, номинальные и максимальные припуски на обработку. Они удаляются с поверхности заготовки в процессе ее обработки для получения детали.

Рассчитаем припуски на механическую обработку для получения размеров штамповки.

Качество поверхности поковки для метода горячей объемной штамповки Rz = 80 мкм, h = 150 мкм ([2], стр. 186, табл. 12).

Габаритный размер детали 125 мм.

Для определения припуска стальных заготовок, изготовляемых методами объемной горячей штамповки, используется зависимость:
$$Z = K_\text{точн} \cdot K_\text{мат} \cdot K_\text{сл} \cdot m_\text{д}^0.1544 \cdot L_H^0.27 \cdot R_\alpha^-0.0238 $$

$K_\text{точн}$ - коэффициент, учитывающий квалитет точности (для данной детали $K_\text{точн}$ = 1);

$K_\text{сл}$ – коэффициент сложности штамповки в зависимости от С1 и С2, в нашем случае $K_\text{сл}$ =1;

$K_\text{мат}$ – коэффициент, учитывающий вид материала заготовки. Наш материал по данной градации относится к М2, следовательно $K_\text{мат}$ = 1,1528;

$m_\text{д}$ – масса детали = 0,35 кг;

$L_H$ – габаритный размер элемента детали;

$R_\alpha $ = 1,6 – шероховатость размерной обработки;










