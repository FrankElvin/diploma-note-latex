\Referat

Расчетно-пояснительная записка содержит \pageref{LastPage} страницы, 66 рисунков, 21 таблицу, 46 источников, 6 приложений.

Ключевые слова: противотанковая управляемая ракета, беспилотный авиационный комплекс, разведывательно-ударный беспилотный летательный аппарат, моделирование траектории полета, твердотопливный ускоритель.

Цель выпускной квалификационной работы – проектирование противотанкового комплекса на базе разведывательного-ударного БПЛА и проработка оценки эффективности применения противотанковой управляемой ракеты (ПТУР), входящей в состав комплекса.

Решаемые задачи:
\begin{itemize}
	\item Формирование облика и структуры комплекса;
	\item Проектирование противотанковой управляемой ракеты;
	\item Разработка конструкторской документации ПТУР;
	\item Создание модели полёта ПТУР, оценка эффективности применения ПТУР по наземным целям;
	\item Анализ технологичности конструкции ПТУР, разработка процесса производства детали;
	\item Анализ опасных и вредных факторов производства и его экологическая экспертиза;
	\item Планирование НИР создания комплекса, расчет сметы затрат.
\end{itemize}

Методы проведения исследования: имитационное моделирование, динамическое моделирование, системотехническое проектирование и конструирование.

Научная новизна состоит в моделировании движения противотанковой управляемой ракеты в ходе оценки эффективности её применения.

Практическая значимость состоит в создании математического аппарата для моделирования движения ПТУР, позволяющего формировать требования к облику БЛА при проектировании беспилотного авиационного комплекса на ранних стадиях разработки, обосновывать требований к летно-техническим характеристикам БЛА и выполнять предварительную оценку эффективности применения летательного аппарата до принятия решения на изготовление опытного образца.

Основные результаты выпускной квалификационной работы:
\begin{itemize}
	\item Сформирован облик и структура беспилотного авиационного комплекса средней дальности;
	\item Спроектирована лёгкая противотанковая управляемая ракета;
	\item Разработана конструкторская документация ПТУР и его двигателя;
	\item Создана модель боевого вылета ПТУР, позволяющая оптимизировать метод наведения ПТУР и оценивать эффективность применения комплекса.
\end{itemize}
