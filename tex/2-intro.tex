\Introduction

Беспилотные авиационные комплексы (БАК) в настоящее время привлекают все большее внимание военно-политических и военно-промышленных кругов во всех странах мира в связи с возможностью решения различного рода задач вооруженной борьбы в постоянно усложняющихся условиях боевых действий с минимальными людскими потерями.


Современные противотанковые управляемые ракеты обладают мощными вычислительными блоками, позволяющими реализовывать манёвр для поражения целей в верхнюю проекцию. Поражать цель в верхнюю проекцию выгодно, так как, во-первых, беспилотный летательный аппарат (БПЛА) сам находится сверху и запускает противотанковую управляемую ракету (ПТУР), находясь на некоторой высоте, и, во-вторых, защищенность бронетехники наиболее низка при поражении в верхнюю проекцию.


В некоторых образцах ПТУР поражение в верхнюю проекцию осуществляется на пролете за счет заряда, установленного перпендикулярно корпусу. Другие ПТУР имеют классическую кумулятивную БЧ и двигаются к цели по крутой траектории.


В данной работе рассматривается возможность использования против бронированных целей, в том числе танков противника, легкой дозвуковой противотанковой управляемой ракеты, поражающей цели в верхнюю проекцию. Также прорабатывается облик комплекса, состоящего из БПЛА-носителя и разрабатываемой управляемой противотанковой ракеты.


Основными задачами противотанковых комплексах на основе БПЛА являются:
\begin{itemize}
	\item Разведыывательная деятельность, обнаружение одиночных целей и объектов и группировок войск противника;
	\item Уточнение имеющейся разведывательной информации о расположении войск и объектов противниа;
	\item Предоставление картинки в реальном времени;
	\item Поражение техники и объектов противника имеющимися на борту средствами;
	\item Целеуказание для других ударных и огневых систем.
\end{itemize}

Второстепеннымми задачами для комплекса являются:
\begin{itemize}
	\item поражение РЭС системы управления войсками и оружием, разведки и РЭБ противника, ведение комплексного технического контроля мероприятий по радиоэлектронной защите;
	\item оценка результатов нанесения ударов;
	\item разведка районов высадки и маршрутов движения разведывательно-диверсионных групп противника (бандформирований);
	\item ретрансляция данных и команд управления;
	\item инженерная, радиационная, химическая и бактериологическая разведка;
	\item доставка грузов в назначенные районы.
\end{item}

Основные исходные данные для проектирования:

\begin{itemize}
	\item Максимальная дальность работы системы обнаружения, км: 10;
	\item Диапазон высот пуска, м: 500 .. 4000;
	\item Тип ПВО противинка: 12,7-мм пулемет (Browning M2 или ДШК);
	\item Бронепробитие, мм по нормали: 800;
	\item Скорость БПЛА при пуске: 150 км/ч.
\end{itemize}


Актуальность разработки комплекса обусловлена:

\begin{itemize}
	\item успешным применением в мире комплексов с БЛА различного целевого назначения в вооруженных конфликтах различного масштаба;
	\item преимуществом, которым располагают БАК по сравнению с пилотируемыми АК – независимостью максимального времени полета, суточного и месячного налётов от физиологических возможностей летного экипажа;
	\item потребностью в современных средствах борьбы (противодействия), соответствующих современным требованиям ведения активных военных действий, с целью уменьшения потерь личного состава;
	\item уникальными боевыми свойствами и возможностью применения в условиях, когда использование пилотируемой авиации невозможно, не эффективно или экономически не выгодно.
\end{itemize}

\clearpage
Дипломный проект состоит из пяти частей.

В конструкторской части определены облик комплекса, его состав и общий принцип функционирования комплекса, приведено описание устройства и состава оборудования БЛА, входящего в этот комплекс. Были выполнены баллистические расчеты стартового и маршевого участков полета управляемой ракеты. В расчетной части был выполнены расчеты боевой части и двигателя ПТУР, а также аэродинамических параметров ПТУР.

В исследовательской части решена задача имитационного моделирования боевого вылета БЛА. Боевой вылет включает в себя полет по маршруту в район предполагаемого расположения цели, выполнение боевой задачи и полет в район посадки. Имитационная модель состоит из трех частей:

В исследовательской части решена задача выбора закона наведения для ПТУР, который позволил бы более рацоинально поражать наземные цели. Также проведена оптимизация этого закона наведения с целью повысить эффективность поражения.

В технологической части дипломного проекта рассматривается разработка технологического процесса производства детали <<Днище двигательной установки>>. Разработана маршрутная карта технологического процесса изготовления детали. Для основных операций разработаны операционные эскизы.

В части <<Экология и промышленная безопасность>> рассмотрены вопросы экологии промышленной безопасности на производстве. Проанализированы опасные и вредные факторы процесса изготовления детали, рассматриваемой в технологической части проекта. Выполнен расчет системы освещения механического цеха, в котором производится изготовление детали.

В организационно-экономической части разработан план-график выполнения опытно-конструкторской работы по созданию комплекса и рассчитана смета затрат на НИР, входящий в тематику по разработке комплекса.

