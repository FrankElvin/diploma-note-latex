\Introduction

Беспилотные авиационные комплексы (БАК) в настоящее время привлекают все большее внимание военно-политических и военно-промышленных кругов во всех странах мира в связи с возможностью решения различного рода задач вооруженной борьбы в постоянно усложняющихся условиях боевых действий с минимальными людскими потерями.


Современные противотанковые управляемые ракеты обладают мощными вычислительными блоками, позволяющими реализовывать манёвр для поражения целей в верхнюю проекцию. Поражать цель в верхнюю проекцию выгодно, так как, во-первых, беспилотный летательный аппарат (БПЛА) сам находится сверху и запускает противотанковую управляемую ракету (ПТУР), находясь на некоторой высоте, и, во-вторых, защищенность бронетехники наиболее низка при поражении в верхнюю проекцию.


В некоторых образцах ПТУР поражение в верхнюю проекцию осуществляется на пролете за счет заряда, установленного перпендикулярно корпусу. Другие ПТУР имеют классическую кумулятивную БЧ и двигаются к цели по крутой траектории.


В данной работе рассматривается возможность использования против бронированных целей, в том числе танков противника, легкой дозвуковой противотанковой управляемой ракеты, поражающей цели в верхнюю проекцию.

